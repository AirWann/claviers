\documentclass[12pt, a4paper]{article}
\usepackage{biblatex} 
\addbibresource{biblio.bib}
\usepackage[T1]{fontenc}    % Encodage des accents
\usepackage[utf8]{inputenc} % Lui aussi
\usepackage[french]{babel} % Pour la traduction française
\usepackage{numprint}       % Histoire que les chiffres soient bien

\usepackage{amsmath}        % La base pour les maths
\usepackage{mathrsfs}       % Quelques symboles supplémentaires
\usepackage{amsfonts}       % Des fontes, eg pour \mathbb.

\usepackage{knowledge}

\usepackage[svgnames]{xcolor} % De la couleur
\usepackage{geometry}       % Gérer correctement la taille


\usepackage{graphicx} % inclusion des graphiques
\usepackage{wrapfig}  % Dessins dans le texte.
\usepackage{mdframed}
\usepackage{thmbox}
\usepackage{tikz}     % Un package pour les dessins (utilisé pour l'environnement {code})
\usepackage{tikz-cd}  % Pour les diagrammes commutatifs
\usetikzlibrary{cd}
\usepackage[toc,page]{appendix}
\usepackage{contour}
\usepackage[normalem]{ulem}
\renewcommand{\ULdepth}{1.8pt}
\contourlength{0.6pt}

%\usepackage{scrhack}
%\usepackage{fix-cm}
\usepackage{mathtools}
\usepackage{amssymb, bm}
%\usepackage{etoolbox}
\usepackage{xparse}
\usepackage{newpxtext,newtxmath}
\usepackage[babel=true]{microtype}
\usepackage{csquotes}

\newcommand*{\retour}{\mathord{\leftarrow}}
\newcommand*{\gauche}{\mathord{\blacktriangleleft}}
\newcommand*{\droite}{\mathord{\blacktriangleright}}
\newcommand*{\entree}{\mathord{\blacksquare}}

\newcommand*{\suppr}{\mathord{\rightarrow}}
\newcommand{\egauche}[1]{\overleftarrow{\mathtt{#1}}}
\newcommand{\edroite}[1]{\overrightarrow{\mathtt{#1}}}


\newcommand*{\Clav}{\ensuremath{\mathsf{Clav}}}
\newcommand*{\Rat}{\ensuremath{\mathsf{Rat}}}
\newcommand*{\Rec}{\ensuremath{\mathsf{Rec}}}
\newcommand*{\Alg}{\ensuremath{\mathsf{Alg}}}
\newcommand*{\Cont}{\ensuremath{\mathsf{Cont}}}

\newcommand*{\MK}{\ensuremath{\mathsf{MK}}}

\newcommand*{\EK}{\ensuremath{\mathsf{EK}}}
\newcommand*{\RK}{\ensuremath{\mathsf{RK}}}
\newcommand*{\REK}{\ensuremath{\mathsf{REK}}}

\newcommand*{\FK}{\ensuremath{\mathsf{FK}}}
\newcommand*{\FEK}{\ensuremath{\mathsf{FEK}}}
\newcommand*{\FRK}{\ensuremath{\mathsf{FRK}}}

\newcommand*{\GK}{\ensuremath{\mathsf{GK}}}
\newcommand*{\GRK}{\ensuremath{\mathsf{GRK}}}
\newcommand*{\GEK}{\ensuremath{\mathsf{GEK}}}

\newcommand*{\FREK}{\ensuremath{\mathsf{FREK}}}
\newcommand*{\GREK}{\ensuremath{\mathsf{GREK}}}
\newcommand*{\BREK}{\ensuremath{\mathsf{BREK}}}

\newcommand*{\QMK}{\ensuremath{\mathsf{QMK}}}
\newcommand*{\QRK}{\ensuremath{\mathsf{QRK}}}
\newcommand*{\QGRK}{\ensuremath{\mathsf{QGRK}}}
\newcommand*{\QFRK}{\ensuremath{\mathsf{QFRK}}}

\newcommand*{\BK}{\ensuremath{\mathsf{BK}}}
\newcommand*{\BRK}{\ensuremath{\mathsf{BRK}}}

\let\touche\mathtt
\let\key\touche

\newcommand*{\conf}[2]{
	\left\langle #1 \middle| #2 \right\rangle
}
\let\config\conf


\newcommand*{\act}[1]{\xrightarrow{#1}}
\newcommand*{\actEff}[1]{\xrightarrow{#1}_{\text{e}}}
\newcommand*{\cdotEff}{\odot} %{\cdot_{\text{e}}}

\newcommand{\ul}[1]{%
	\uline{\phantom{#1}}%
	\llap{\contour{white}{#1}}%
}
\hyphenpenalty 500

\newcommand{\Z}{\mathbb{Z}} 
\newcommand{\Zp}[1][p]{\Z/#1\Z} 
\newcommand{\newlemme}[1]{\newtheorem[S]{#1}[cpt]{Lemme}}
\newcommand{\newdef}[1]{\newtheorem[S]{#1}[cpt]{Définition}}
\newcommand{\newthm}[1]{\newtheorem[M]{#1}[cpt]{Théorème}}
\renewcommand{\L}{\mathcal{L}}
\newcounter{cpt}

\newdef{defétats}
\newdef{configétats}
\newlemme{RatdansQRK}
\newdef{utiles}
\newlemme{inutilesbornés}
\newthm{QRKegalRat}

\begin{document}
    \section{Claviers}




	\section{Claviers à états}
    Les claviers comme définis précédemment\cite{bible} n'ont aucune mémoire, ce qui les rend incapables de stocker une quelconque information.
    Pour pallier à cela, on ajoute des états :
    \begin{defétats}[Clavier à état]
        Un clavier à état est la donnée de $(Q,K,\Delta ,F,s_0)$ où :
        \begin{itemize}
            \item $Q$ est un ensemble d'\textbf{états}
            \item $K$ est un clavier
            \item $\Delta \subset Q \times K \times Q$ est un ensemble fini de \textbf{transitions}
            \item $F$ est un ensemble d'états dits \textbf{finaux}
            \item $s_0 \in Q$ est un état dit \textbf{initial}
        \end{itemize}
    \end{defétats}
    \begin{configétats}[Configuration étatique]
        L'état dans lequel se trouve un clavier à états durant son calcul peut être décrit par un triplet $(u,q,v) \in A^* \times Q \times A^*$, que l'on appelle \textbf{configuration étatique}.

    \end{configétats}
    Intuitivement, l'idée d'un clavier à états est que seules certaines touches sont applicables depuis un certain état. L'application d'une touche modifie le mot (comme pour les claviers habituels)
    et fait passer dans un autre état (comme pour un automate).

    On notera $\mathsf{Q}X$ l'ensemble des claviers à états sur des claviers de classe $X$ : $\QMK$, $\QFRK$, etc... 
    On négligera la présence du symbole entrée, car on accepte par état final.
    Par abus de langage, on dira d'un langage $\L$ qu'il est dans une classe pour dire qu'il est reconnu par un clavier de cette classe : 
    par exemple on dira $\L \in \QGRK$ si il existe $K \in \QGRK$ tel que $\L = \L(K)$.


    \subsection{QRK = Rat}
    On sait que (théorème 101/102) $\RK \subset \Rat$, et que l'inclusion est stricte, comme le montre le langage $a^* + b^*$. En ajoutant des états,
    $\QRK$ est donc un bon candidat pour être égal à $\Rat$, et la preuve de cette égalité est l'objet de cette section.
    
    \begin{RatdansQRK}
        $\Rat \subset \QRK$
    \end{RatdansQRK}
    En effet, étant donné un automate $\mathcal{A}$, on peut le voir comme un clavier à états, dont les touches étiquetant les transitions 
    ne font que écrire une unique lettre.

    Pour pouvoir montrer l'inclusion inverse, on va d'abord définir les notions de lettre utile et inutile.
    \begin{utiles}
            yyyyyyy
    \end{utiles}    
    \begin{inutilesbornés}
        Soit $K$ un clavier à états de $\QRK$. Il existe $\mathcal{B}(K)$, ne dépendant que de $K$, telle que pout tout mot $\omega \in \L(K)$,
        il existe une exécution de $K$ qui écrit $\omega$ et qui utilise au plus $\mathcal{B}(K)$ lettres inutiles.
    \end{inutilesbornés}


    \clearpage
    \printbibliography
\end{document}
