\documentclass[12pt, a4paper]{article}
\usepackage{biblatex} 
\addbibresource{biblio.bib}
\usepackage[T1]{fontenc}    % Encodage des accents
\usepackage[utf8]{inputenc} % Lui aussi
\usepackage[french]{babel} % Pour la traduction française
\usepackage{numprint}       % Histoire que les chiffres soient bien

\usepackage{amsmath}        % La base pour les maths
\usepackage{mathrsfs}       % Quelques symboles supplémentaires
\usepackage{amsfonts}       % Des fontes, eg pour \mathbb.

\usepackage{knowledge}

\usepackage[svgnames]{xcolor} % De la couleur
\usepackage{geometry}       % Gérer correctement la taille


\usepackage{graphicx} % inclusion des graphiques
\usepackage{wrapfig}  % Dessins dans le texte.
\usepackage{mdframed}
\usepackage{thmbox}
\usepackage{tikz}     % Un package pour les dessins
\usepackage{tikz-cd}
\usetikzlibrary{cd, automata,positioning,arrows}
\usepackage[toc,page]{appendix}
\usepackage{contour}
\usepackage[normalem]{ulem}
\renewcommand{\ULdepth}{1.8pt}
\contourlength{0.6pt}
\usepackage{float}
%\usepackage{scrhack}
%\usepackage{fix-cm}
\usepackage{mathtools}
\usepackage{amssymb, bm}
%\usepackage{etoolbox}
\usepackage{xparse}
\usepackage{newpxtext,newtxmath}
\usepackage[babel=true]{microtype}
\usepackage{csquotes}
\usepackage{stmaryrd}
\newcommand*{\retour}{\mathord{\leftarrow}}
\newcommand*{\gauche}{\mathord{\blacktriangleleft}}
\newcommand*{\droite}{\mathord{\blacktriangleright}}
\newcommand*{\entree}{\mathord{\blacksquare}}

\newcommand*{\suppr}{\mathord{\rightarrow}}
\newcommand{\egauche}[1]{\overleftarrow{\mathtt{#1}}}
\newcommand{\edroite}[1]{\overrightarrow{\mathtt{#1}}}


\newcommand*{\Clav}{\ensuremath{\mathsf{Clav}}}
\newcommand*{\Rat}{\ensuremath{\mathsf{Rat}}}
\newcommand*{\Rec}{\ensuremath{\mathsf{Rec}}}
\newcommand*{\Alg}{\ensuremath{\mathsf{Alg}}}
\newcommand*{\Cont}{\ensuremath{\mathsf{Cont}}}

\newcommand*{\MK}{\ensuremath{\mathsf{MK}}}

\newcommand*{\EK}{\ensuremath{\mathsf{EK}}}
\newcommand*{\RK}{\ensuremath{\mathsf{RK}}}
\newcommand*{\REK}{\ensuremath{\mathsf{REK}}}

\newcommand*{\FK}{\ensuremath{\mathsf{FK}}}
\newcommand*{\FEK}{\ensuremath{\mathsf{FEK}}}
\newcommand*{\FRK}{\ensuremath{\mathsf{FRK}}}

\newcommand*{\GK}{\ensuremath{\mathsf{GK}}}
\newcommand*{\GRK}{\ensuremath{\mathsf{GRK}}}
\newcommand*{\GEK}{\ensuremath{\mathsf{GEK}}}

\newcommand*{\FREK}{\ensuremath{\mathsf{FREK}}}
\newcommand*{\GREK}{\ensuremath{\mathsf{GREK}}}
\newcommand*{\BREK}{\ensuremath{\mathsf{BREK}}}

\newcommand*{\QMK}{\ensuremath{\mathsf{QMK}}}
\newcommand*{\QRK}{\ensuremath{\mathsf{QRK}}}
\newcommand*{\QGRK}{\ensuremath{\mathsf{QGRK}}}
\newcommand*{\QFRK}{\ensuremath{\mathsf{QFRK}}}

\newcommand*{\BK}{\ensuremath{\mathsf{BK}}}
\newcommand*{\BRK}{\ensuremath{\mathsf{BRK}}}

\let\touche\mathtt
\let\key\touche

\newcommand*{\conf}[2]{
	\left\langle #1 \middle| #2 \right\rangle
}
\let\config\conf


\newcommand*{\act}[1]{\xrightarrow{#1}}
\newcommand*{\actEff}[1]{\xrightarrow{#1}_{\text{e}}}
\newcommand*{\cdotEff}{\odot} %{\cdot_{\text{e}}}

\newcommand{\ul}[1]{%
	\uline{\phantom{#1}}%
	\llap{\contour{white}{#1}}%
}
\hyphenpenalty 500

\newcommand{\Z}{\mathbb{Z}} 
\newcommand{\Zp}[1][p]{\Z/#1\Z} 
\newcommand{\newlemme}[1]{\newtheorem[S]{#1}[cpt]{Lemme}}
\newcommand{\newdef}[1]{\newtheorem[S]{#1}[cpt]{Définition}}
\newcommand{\newthm}[1]{\newtheorem[M]{#1}[cpt]{Théorème}}
\renewcommand{\L}{\mathcal{L}}
\newcommand{\A}{\mathcal{A}}
\newcounter{cpt}

\newdef{defétats}
\newdef{configétats}
\newlemme{RatdansQRK}
\newdef{utiles}
\newlemme{inutilesbornés}
\newthm{QRKegalRat}

\begin{document}
    \section{Claviers}




	\section{Claviers à états}
    Les claviers comme définis précédemment n'ont aucune mémoire, ce qui les rend incapables de stocker une quelconque information.
    Pour pallier à cela, on ajoute des états :
    \begin{defétats}[Clavier à état]
        Un clavier à état est la donnée de $(Q,K,\Delta ,F,s_0)$ où :
        \begin{itemize}
            \item $Q$ est un ensemble d'\textbf{états}
            \item $K$ est un clavier
            \item $\Delta \subset Q \times K \times Q$ est un ensemble fini de \textbf{transitions}
            \item $F$ est un ensemble d'états dits \textbf{finaux}
            \item $s_0 \in Q$ est un état dit \textbf{initial}
        \end{itemize}
    \end{defétats}
    \begin{configétats}[Configuration étatique]
        L'état dans lequel se trouve un clavier à états durant son calcul peut être décrit par un triplet $(u,q,v) \in A^* \times Q \times A^*$, que l'on appelle \textbf{configuration étatique}.

    \end{configétats}
    Intuitivement, l'idée d'un clavier à états est que seules certaines touches sont applicables depuis un certain état. L'application d'une touche modifie le mot (comme pour les claviers habituels)
    et fait passer dans un autre état (comme pour un automate).

    On notera $\mathsf{Q}X$ l'ensemble des claviers à états sur des claviers de classe $X$ : $\QMK$, $\QFRK$, etc... 
    On négligera la présence du symbole entrée, car on accepte par état final.
    Par abus de langage, on dira d'un langage $\L$ qu'il est dans une classe pour dire qu'il est reconnu par un clavier de cette classe : 
    par exemple on dira $\L \in \QGRK$ si il existe $K \in \QGRK$ tel que $\L = \L(K)$.


    \subsection{QRK = Rat}
    On sait que \autocite[théorèmes~101/102]{bible} $\RK \subset \Rat$, et que l'inclusion est stricte, comme le montre le langage $a^* + b^*$. En ajoutant des états,
    $\QRK$ est donc un bon candidat pour être égal à $\Rat$, et la preuve de cette égalité est l'objet de cette section.
    
    \begin{RatdansQRK}
        $\Rat \subset \QRK$
    \end{RatdansQRK}
    En effet, étant donné un automate $\A$, on peut le voir comme un clavier à états, dont les touches étiquetant les transitions 
    ne font que écrire une unique lettre.

    Pour pouvoir montrer l'inclusion inverse, on va étudier plus précisément la structure du mot qu'on est en train d'écrire\footnote{On considère les automates comme \emph{générateurs}, qui écrivent progressivement le mot, plutôt que comme acceptateurs, qui le lisent puis acceptent ou non.}.
    
    Soit $K$ un clavier de $\QRK$. En l'absence de flèche gauche, on sait que toutes les configurations étatiques sont de la forme $(w,q,\varepsilon)$. De plus, lors d'un calcul acceptant, on peut à chaque étape diviser le mot courant $w$ en deux :
    \begin{itemize}
        \item un préfixe $u$ qui ne sera jamais effacé, qui est donc un préfixe du mot qu'on cherche à écrire ; on dit qu'il est \emph{utile}.
        \item un suffixe $x$ qui sera effacé par les opérations à suivre ; on dit qu'il est \emph{inutile}.
    \end{itemize}
    On montre alors
    \begin{inutilesbornés}
        Soit $K$ un clavier à états de $\QRK$. Il existe $\mathcal{B}(K)$, ne dépendant que de $K$, telle que pout tout mot $w \in \L(K)$,
        il existe une exécution de $K$ qui écrit $w$ et dont le suffixe inutile est de longueur majorée par $\mathcal{B}(K)$ à chaque étape.
    \end{inutilesbornés}
    L'idée pour construire un automate reconnaissant le même langage que $K$ est donc d'utiliser comme états $Q \times \llbracket 0 ; \mathcal{B}(K) \rrbracket$, dans lesquels on stocke dans la première composante l'état du clavier dans lequel on se trouve,
    et dans la deuxième composante la taille du suffixe inutile. Notre automate n'écrit une lettre que quand notre clavier écrit une lettre \emph{utile}.
    On rappelle que toute touche $t$ d'un clavier sans flèche est équivalente à $\retour^r w$ pour un certain $r$ et un certain $w$, qui peuvent être calculés en temps polynomial en la taille de $t$. On suppose que cette transformation a été effectuée et on note alors $\sharp t = |w| - r$

    Formellement, on procède comme cela : 
    \begin{itemize}
        \item on rajoute un état $\iota$ à notre clavier, et on rajoute, pour toute transition $(s_{init},\retour^n w,s_{\alpha})$ une transition $(\iota,w,s_{\alpha})$
        \item on considère $\A$ dont les états sont  $( Q \times \llbracket 0 ; \mathcal{B}(K) \rrbracket ) \cup i$
        \item pour toute transition $s_{\alpha} \xrightarrow{\retour^r w} s_{\beta}$ de notre clavier à états, on crée dans $\A$ les transitions suivantes (voir figure \ref{ok}):
            \begin{itemize}
                \item pour tout $k$ tel que $r \leqslant k \leqslant \mathcal{B}(K) - \sharp t$, $s_{\alpha, k} \xrightarrow{\varepsilon} s_{\beta, k + \sharp t}$
                \item pour tout $j$ tel que $0 \leqslant j \leqslant |w| - 1$, $s_{\alpha, r} \xrightarrow{w[1,|w|-j]} s_{\beta, j}$
            \end{itemize} 
        \item pour toute transition $\iota \xrightarrow{w} s_{\beta}$ du clavier, on ajoute, pour tout $j$ tel que $0 \leqslant j \leqslant |w|$, $i \xrightarrow{w[1,|w|-j]} s_{\beta, j}$ (avec la convention $w[1,0] = \varepsilon$)   
    \end{itemize}
    \begin{figure}[!ht]
        \centering
        \begin{tikzcd}
            i \arrow[rrr, "w"] \arrow[rrrddd, "\varepsilon"] \arrow[rrrd, "{w[1,j]}", dotted] &  &  & {s_{\beta,0}}   &  & {s_{\alpha,0}}                                                                                                                     &  &  & {s_{\beta,0}}              \\
                                                                                              &  &  & \vdots          &  & \vdots                                                                                                                             &  &  & \vdots                     \\
                                                                                              &  &  & \vdots          &  & {s_{\alpha,r}} \arrow[rrruu, "w", bend left] \arrow[rrru, "{w[1,|w|-1]}"] \arrow[rrr, "\dots", dotted] \arrow[rrrd, "\varepsilon"] &  &  & {}                         \\
                                                                                              &  &  & {s_{\beta,|w|}} &  & {s_{\alpha,r+1}} \arrow[rrrd, "\varepsilon"]                                                                                       &  &  & {s_{\beta,|w|}}            \\
                                                                                              &  &  & \vdots          &  & \vdots \arrow[rrrd, "\varepsilon", dotted]                                                                                         &  &  & \vdots                     \\
                                                                                              &  &  &                 &  & \vdots \arrow[rrrd, "\varepsilon", dotted]                                                                                         &  &  & \vdots                     \\
                                                                                              &  &  &                 &  & {s_{\alpha,\mathcal{B}(K)}}                                                                                                        &  &  & {s_{\beta,\mathcal{B}(K)}}
        \end{tikzcd}
        \caption{Les transitions ajoutées}
        \label{ok}
    \end{figure}

    \clearpage
    \printbibliography
\end{document}
