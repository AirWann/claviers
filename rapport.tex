\documentclass[12pt, a4paper]{article}
\usepackage{biblatex} 
\addbibresource{biblio.bib}
\usepackage[T1]{fontenc}    % Encodage des accents
\usepackage{lmodern}
\usepackage[utf8]{inputenc} % Lui aussi
\usepackage[french]{babel} % Pour la traduction française
\usepackage{numprint}       % Histoire que les chiffres soient bien

\usepackage{amsmath}        % La base pour les maths
%\usepackage{mathrsfs}       % Quelques symboles supplémentaires
%\usepackage{amsfonts}       % Des fontes, eg pour \mathbb.


\usepackage[svgnames]{xcolor} % De la couleur
\usepackage{geometry}       % Gérer correctement la taille

\setlength{\marginparwidth}{2cm}
\usepackage{todonotes}
\usepackage{graphicx} % inclusion des graphiques
\usepackage{wrapfig}  % Dessins dans le texte.
\usepackage{mdframed}
\usepackage{thmbox}
\usepackage{tikz}     % Un package pour les dessins
\usepackage{tikz-cd}
\usetikzlibrary{cd, automata,positioning,arrows}
\usepackage[toc,page]{appendix}
\usepackage{contour}
\usepackage[normalem]{ulem}
\renewcommand{\ULdepth}{1.8pt}
\contourlength{0.6pt}
\usepackage{float}
%\usepackage{scrhack}
%\usepackage{fix-cm}
\usepackage{mathtools}
\usepackage{amssymb, bm}
%\usepackage{etoolbox}
\usepackage{xparse}
\usepackage{newpxtext,newtxmath}
\usepackage[babel=true]{microtype}
\usepackage{csquotes}
\usepackage{stmaryrd}
\usepackage{adjustbox}
\newcommand*{\retour}{\mathord{\leftarrow}}
\newcommand*{\gauche}{\mathord{\blacktriangleleft}}
\newcommand*{\droite}{\mathord{\blacktriangleright}}
\newcommand*{\entree}{\mathord{\blacksquare}}

\newcommand*{\suppr}{\mathord{\rightarrow}}
\newcommand{\egauche}[1]{\overleftarrow{\mathtt{#1}}}
\newcommand{\edroite}[1]{\overrightarrow{\mathtt{#1}}}


\newcommand*{\Clav}{\ensuremath{\mathsf{Clav}}}
\newcommand*{\Rat}{\ensuremath{\mathsf{Rat}}}
\newcommand*{\Rec}{\ensuremath{\mathsf{Rec}}}
\newcommand*{\Alg}{\ensuremath{\mathsf{Alg}}}
\newcommand*{\Cont}{\ensuremath{\mathsf{Cont}}}

\newcommand*{\MK}{\ensuremath{\mathsf{MK}}}

\newcommand*{\EK}{\ensuremath{\mathsf{EK}}}
\newcommand*{\RK}{\ensuremath{\mathsf{RK}}}
\newcommand*{\REK}{\ensuremath{\mathsf{REK}}}

\newcommand*{\FK}{\ensuremath{\mathsf{FK}}}
\newcommand*{\FEK}{\ensuremath{\mathsf{FEK}}}
\newcommand*{\FRK}{\ensuremath{\mathsf{FRK}}}

\newcommand*{\GK}{\ensuremath{\mathsf{GK}}}
\newcommand*{\GRK}{\ensuremath{\mathsf{GRK}}}
\newcommand*{\GEK}{\ensuremath{\mathsf{GEK}}}

\newcommand*{\FREK}{\ensuremath{\mathsf{FREK}}}
\newcommand*{\GREK}{\ensuremath{\mathsf{GREK}}}
\newcommand*{\BREK}{\ensuremath{\mathsf{BREK}}}

\newcommand*{\QMK}{\ensuremath{\mathsf{QMK}}}
\newcommand*{\QRK}{\ensuremath{\mathsf{QRK}}}
\newcommand*{\QGRK}{\ensuremath{\mathsf{QGRK}}}
\newcommand*{\QFRK}{\ensuremath{\mathsf{QFRK}}}

\newcommand*{\BK}{\ensuremath{\mathsf{BK}}}
\newcommand*{\BRK}{\ensuremath{\mathsf{BRK}}}

\let\touche\mathtt
\let\key\touche

\newcommand*{\conf}[2]{
	\left\langle #1 \middle| #2 \right\rangle
}
\let\config\conf


\newcommand*{\act}[1]{\xrightarrow{#1}}
\newcommand*{\actEff}[1]{\xrightarrow{#1}_{\text{e}}}
\newcommand*{\cdotEff}{\odot} %{\cdot_{\text{e}}}

\newcommand{\ul}[1]{%
	\uline{\phantom{#1}}%
	\llap{\contour{white}{#1}}%
}
\hyphenpenalty 500

\newcommand{\newlemme}[1]{\newtheorem[S]{#1}[cpt]{Lemme}}
\newcommand{\newdef}[1]{\newtheorem[S]{#1}[cpt]{Définition}}
\newcommand{\newthm}[1]{\newtheorem[M]{#1}[thm]{Théorème}}
\newcommand{\newcor}[1]{\newtheorem[S]{#1}[cpt]{Corollaire}}
\renewcommand{\L}{\mathcal{L}}
\renewcommand{\bar}{\overline}
\newcommand{\A}{\mathcal{A}}
\newcommand{\Kinf}{||K||_{\infty}}
\newcounter{cpt}
\newcounter{thm}

\newdef{semopel}
\newdef{touche}
\newdef{clavier}
\newdef{defétats}
\newdef{configétats}
\newdef{tailleclavierQ}
\newlemme{RatdansQRK}
\newlemme{pas}
\newdef{utiles}
\newlemme{inutilesbornés}
\newlemme{bisimautomclav}
\newthm{QRKegalRat}
\newcor{motQRK}
\newdef{defopsym}
\newdef{defclavsym}


\newcommand{\corto}[1]{\todo[color=blue!30]{\small #1}}
\newcommand{\cortoin}[1]{\todo[color=blue!30,inline]{\small #1}}
\newcommand{\vincent}[1]{\textcolor{blue}{#1 -- Vincent}}
\newcommand{\erwann}[1]{\textcolor{green}{Erwann : #1}}

\title{Claviers symétriques et à états}
\author{Erwann LOULERGUE, \\ supervisé par Vincent PENELLE et Corto MASCLE}
\date{Juin \& Juillet 2022}
\begin{document}
    \maketitle
    Ce stage suit un article de 2021, \cite{bible}, introduisant un nouveau modèle de calcul,
    baptisé claviers.
    Informellement, ce modèle est un système de réécriture de mots inspiré par le comportement du clavier d’un ordinateur, mais dont les touches appliqueraient plusieurs opérations basiques (qui sont celles que peut effectuer un clavier habituel). Les opérations sont appliquées depuis la position d’un curseur et sont constituées de d’écritures de lettres, d’effacement indiscriminé de lettres et de déplacement de curseurs, similaires aux opérations que peut effectuer le clavier d’un ordinateur. En fonctions des opérations de bases autorisées, on obtient différents modèles de claviers qui engendrent a priori différentes classes de langages
    L’un des résultats de celui-ci est que le langage de tout clavier
    utilisant seulement l’écriture de lettres et l’effacement à gauche est rationnel. 
    On considère différents ensembles d’opérations de bases pour étudier l’expressivité des modèles obtenus.
    
    Le stage se divise en deux parties, qui correspondent à deux extensions du modèle des claviers, chacun étendant le modèle "de base" avec quelque chose qu'il n'a pas :
    \begin{itemize}
        \item les claviers à états, visant à ajouter de la  \emph{mémoire}
        \item les claviers symétriques, ajoutant de la \emph{symétrie}.
    \end{itemize}
    
    Chacun de ces modèles a occupé environ la moitié du stage\footnote{Le sujet annoncé était plus centré sur les claviers symétriques, mais l'étude des claviers à états a finalement pris une part importante.}. 
    La question principale des claviers à états est résolue (voir section \ref{sectionclavetats}), mais celle des claviers symétriques reste une conjecture (voir section \ref{sectionclavsym})
    \clearpage    
    
    
    \emph{Dans tout le document, $A$ désigne un alphabet quelconque. On convient que les mots sont indexés à partir de 1, et que $w[i,j]$ désigne les lettres de la $i$-ème à la $j$-ème incluses.}
    \section{Claviers}

    
    Nous considérons l’ensemble des \emph{opérations élémentaires} $S = A \cup \{\retour, \gauche, \droite\}$, où $\retour$ est appelé \emph{retour arrière}, et $\gauche$, $\droite$, \emph{flèche gauche, flèche droite}.
    Une configuration est un élément de $A^* \times A^*$, noté $\conf{u}{v}$. Intuitivement, dans cette configuration, il y a $uv$ dans la zone de texte, avec le curseur entre les deux.
    On peut alors définir la sémantique des opérations élémentaires :
    \begin{semopel}
        L'\textbf{action (élémentaire)} $\conf{u}{v}\cdot s$ d'une opération $s$ sur une configuration $\conf{u}{v}$ est définie comme suit :
        \begin{align*}
            \forall a \in A&, \conf{u}{v}\cdot a = \conf{ua}{v} \\
            \conf{\varepsilon}{v}\cdot \retour = \conf{\varepsilon}{v} &, \conf{ua}{v}\cdot \retour = \conf{u}{v} \\
            \conf{\varepsilon}{v}\cdot \gauche = \conf{\varepsilon}{v} &, \conf{ua}{v}\cdot \gauche = \conf{u}{av} \\
            \conf{u}{\varepsilon}\cdot \droite = \conf{u}{\varepsilon} &, \conf{u}{av}\cdot \droite = \conf{ua}{v} \\
        \end{align*}
        Une action élémentaire est dite efficiente si la configuration produite est différente de la configuration à laquelle l’opération est appliquée.
    \end{semopel}
    \begin{touche}[Touches]
        Une \textbf{touche} est un élément de $S^*$, donc un mot sur les opérations élémentaires. \\
    \end{touche}
    La taille $|t|$ d'une touche est sa longueur en tant que mot. Son action est définie inductivement :

    $\begin{cases}
        \conf{u}{v}\cdot \varepsilon = \conf{u}{v} \\
        \conf{u}{v}\cdot st = (\conf{u}{v}\cdot s) \cdot t
    \end{cases}$
    
    Elle est dite efficiente si elle ne contient pas d'action élémentaire non efficiente\footnote{Cela dépend donc du contexte. Une touche donnée n'est pas efficiente ou non efficiente, mais une \emph{action} l'est.}.
    On notera aussi $\conf{u}{v} \rightarrow \conf{u'}{v'}$ le fait qu'il existe $t$ telle que $\conf{u}{v}\cdot t = \conf{u'}{v'}$.
     
    Un ensemble fini de touches est appele \emph{clavier automatique}. Une exécution d'un clavier automatique est une suite finie de touches de ce clavier, et on dit que $w \in A^*$ est \emph{reconnu par $K$} si il existe une exécution de $K$, $t_1\dots t_n$ telle que $\conf{\varepsilon}{\varepsilon}\cdot t_1\dots t_n = \conf{u}{v}$ avec $uv = w$.
    
    Ces claviers sont assez limités : on n'a pas de notion de saisie qu'on choisirait d'arrêter. On définit donc les \emph{claviers manuels} :
    \begin{clavier}[Clavier]
        Un clavier manuel, ou simplement \textbf{clavier}, est un couple $(T,F)$ d'ensembles finis de touches. $T$ sont les touches transientes, et $F$ les touches finales.
        Une exécution acceptante d'un clavier est une suite de touches de $T$ suivie d'une touche de $F$.
    \end{clavier}
    
    Finalement, pour les claviers automatiques, on a \[\L(K) = \{w | w =uv, \exists t_1...t_n, \conf{\varepsilon}{\varepsilon}\cdot t_1...t_n = \conf{u}{v}\}\]
    Sinon, dans le cas général, on définit $\L(K)$ comme \[\{w | w =uv, \exists t_1...t_n, \conf{\varepsilon}{\varepsilon}\cdot t_1...t_n = \conf{u}{v}, \mathrm{seule~} t_n \mathrm{~est~finale}\}\]
    
    \begin{example}[Remarque]
        On peut voir les touches acceptantes comme des touches finissant par un symbole "entrée", $\entree$. Cela nous permet de noter les claviers comme un seul ensemble (par exemple le clavier $(\{aa\},\{a\})$ sera noté $\{aa,a\entree\}$ (ce clavier reconnaît $(aa)^*a$)).   
        L'entrée peut alors être vue comme un symbole qui n'apparaît qu'à la fin des touches et qui termine la saisie. Les claviers automatiques, la saisie n'a pas de fin : si on peut avoir écrit un mot après l'exécution d'un certain nombres de touches, ce mot est reconnu par le clavier.
    \end{example}

    La \emph{taille} d'un clavier $K$ est définie comme $\sum_{t \in K} (|t| + 1)$.
    On note $\Kinf$ la quantité $\max\limits_{t \in K} |t|$.
    
    Les symboles autres que l'écriture de lettres modifient l'expressivité des claviers. On va donc définir différentes classes, qui auront différentes propriétés : toutes autorisent l'écriture de lettres, mais seulement certains caractère spéciaux. On les nomme selon ceux autorisés :
    \[
    \begin{array}{ccc}
        \MK : \{\} & \GK : \{\gauche\} & \FK :\{\gauche,\droite\} \\
        \EK : \{\entree\} & \GEK : \{\gauche,\entree\} & \FEK :\{\gauche,\droite,\entree\} \\
        \RK : \{\retour\} & \GRK : \{\gauche,\retour\} & \FRK :\{\gauche,\droite,\retour\} \\
        \REK : \{\retour, \entree\} & \GREK : \{\gauche,\retour,\entree\} & \FREK :\{\gauche,\droite,\retour,\entree\} \\
    \end{array}
    \]
    On dira, abusivement, qu'un langage est dans une classe s'il existe un clavier de cette classe le reconnaissant. Par exemple, on dira $L \in \GRK$ s'il existe $K \in \GRK$ tel que $\L(K) = L$.
    \clearpage


	\section{Claviers à états}
    Les claviers comme définis précédemment n'ont aucune \emph{mémoire}, ce qui les rend incapables de stocker une quelconque information.
    Une question naturelle consiste à se demander si l’ajout d’états de contrôle modifie l’expressivité du modèle obtenu.
    \begin{defétats}[Clavier à états]
        Un clavier à états est la donnée de $(Q,\Delta ,F,s_0)$ où :
        \begin{itemize}
            \item $Q$ est un ensemble d'\textbf{états}
            \item $\Delta \subset Q \times S^* \times Q$ est un ensemble fini de \textbf{transitions}
            \item $F$ est un ensemble d'états dits \textbf{finaux}
            \item $s_0 \in Q$ est un état dit \textbf{initial}
        \end{itemize}
    \end{defétats}
    \begin{configétats}[Configuration étatique]
        La configuration dans laquelle se trouve un clavier à états à un moment de son calcul peut être décrite par un triplet $(u,q,v) \in A^* \times Q \times A^*$, que l'on appelle \textbf{configuration étatique}.
    \end{configétats}
    
    Soit $(u,q,v)$ une configuration étatique, et $(q,t,q') \in \Delta$. En appliquant cette transition à cette configuration, on passe dans la configuration $(u',q',v')$, avec $\conf{u'}{v'} = \conf{u}{v} \cdot t$.
    On note cette opération $(u,q,v) \cdot (q,t,q') = (u',q',v')$.

    Intuitivement, l'idée d'un clavier à états est que seules certaines touches sont applicables depuis un certain état. L'application d'une touche modifie le mot (comme pour les claviers habituels)
    et fait passer dans un autre état (comme pour un automate).
    On peut donc définir le langage d'un clavier à états :

    Pour $K$ un clavier à états, on définit $\L(K)$ le \textbf{langage de $K$} comme :
    \[ \{uv | \exists\delta_1,...,\delta_n \in \Delta, \exists s_f \in F, (u,s_f,v) = (\varepsilon,s_0,\varepsilon)\cdot\delta_1\cdots\delta_n \}\]

    On notera $\mathsf{Q}X$ l'ensemble des claviers à états dont les transitions sont toutes de classe $X$ : $\QMK$, $\QFRK$, etc... 
    On négligera la présence du symbole entrée, car on accepte par état final (on peut en fait simuler la touche entrée en ajoutants des états, voir annexe).
    Par abus de langage, on dira d'un langage $\L$ qu'il est dans une classe pour dire qu'il est reconnu par un clavier de cette classe : 
    par exemple on dira $\L \in \QGRK$ si il existe $K \in \QGRK$ tel que $\L = \L(K)$.
    \begin{tailleclavierQ}[taille d'un clavier à états]
        Soit $K$ un clavier à états. On note $|K|$, et on appelle \textbf{taille} de $K$, la quantité 
        \[ |Q| + |T| + \sum_{t \in T} |t|\]
    \end{tailleclavierQ}

    \subsection{QRK = Rat}\label{sectionclavetats}
    On sait que $\RK \subset \Rat$ \autocite[théorèmes~101/102]{bible}, et que l'inclusion est stricte, comme le montre le langage $a^* + b^*$. En ajoutant des états,
    $\QRK$ est donc un bon candidat pour être égal à $\Rat$ ; on prouve cette égalité dans cette section.
    
    \begin{RatdansQRK}
        $\Rat \subset \QRK$
    \end{RatdansQRK}
    En effet, étant donné un automate $\A$, on peut le voir comme un clavier à états, dont les touches étiquetant les transitions 
    ne font qu'écrire une seule lettre.
    \medskip

    Pour pouvoir montrer l'inclusion inverse, on va étudier plus précisément la structure du mot qu'on est en train d'écrire avec notre clavier.
    
    Soit $K$ un clavier de $\QRK$. En l'absence de flèche gauche, on sait que toutes les configurations étatiques sont de la forme $(w,q,\varepsilon)$. De plus, lors d'un calcul acceptant, on peut à chaque étape diviser le mot courant $w$ en deux :
    \begin{itemize}
        \item un préfixe $u$ qui ne sera jamais effacé, qui est donc un préfixe du mot qu'on cherche à écrire ; on dit qu'il est \emph{utile}.
        \item un suffixe $x$ qui sera effacé par les opérations à suivre ; on dit qu'il est \emph{inutile}.
    \end{itemize}
    \begin{example}[Remarque]
        Attention : une lettre peut être placée au bon endroit, et ne pas être utile pour autant. Considérons par exemple le clavier à un seul état dont les touches sont $\{abc,\retour^2b\}$.
        L'exécution écrivant $ababc$ est : $\varepsilon \rightarrow abc \rightarrow ab \rightarrow ababc$, mais le $b$ du deuxième mot n'est \emph{pas} utile, car il est effacé à l'étape suivante (même s'il est remplacé par un autre $b$). \smallskip
    \end{example}
    

    On montre alors :
    \begin{inutilesbornés}\label{bk}
        Soit $K$ un clavier à états de $\QRK$. Il existe un entier $\mathcal{B}(K)$, ne dépendant que de $K$, tel que pour tout mot $w \in \L(K)$,
        il existe une exécution de $K$ qui écrit $w$ et dont le suffixe inutile est de longueur majorée par $\mathcal{B}(K)$ à chaque étape. \\
        De plus, $\mathcal{B}(K)$ est polynomial en la taille de $K$.
    \end{inutilesbornés}
    L'idée pour construire un automate reconnaissant le même langage que $K$ est d'utiliser comme états $Q \times \llbracket 0 ; \mathcal{B}(K) \rrbracket$, dans lesquels on stocke dans la première composante l'état du clavier dans lequel on se trouve,
    et dans la deuxième composante la taille du suffixe inutile. Notre automate ne simule pas tous les calculs possibles du clavier, seulement ceux dont le suffixe  inutile est borné par $\mathcal{B}(K)$. Mais le lemme précédent nous garantit que cela suffit à reconnaitre le même langagae. Notre automate ne lit une lettre que quand notre clavier écrit une lettre \emph{utile}.
    
    On rappelle que \autocite[Lemme~74]{bible} toute touche $t$ d'un clavier sans flèche est équivalente à $\retour^r w$ pour un certain $r$ et un certain $w$, qui peuvent être calculés en temps polynomial en la taille de $t$. Sans perte de généralité, on suppose que toutes les touches de notre clavier sont de cette forme, et on note alors $\sharp t = |w| - r$.
    
    On crée cet automate comme suit\footnote{on s'autorise à avoir des transitions dans l'automate étiquetées par des mots. On peut facilement en extraire un automate étiqueté par des lettres.} :
        \begin{itemize}
        % inutile grâce à s_0,0BARRE \item on rajoute un état $init$ à notre clavier, et on rajoute, pour toute transition $(s_0,\retour^n w,s_{\alpha})$ une transition $(init,w,s_{\alpha})$ (\emph{pré-traitement initial})
        \item on considère $\A$ dont les états sont les états de $K$ dupliqués $2(\mathcal{B}(K) + 1)$ fois : pour $\alpha \in Q(K)$, on les note $s_{\alpha, i}$ et $\overline{s_{\alpha, i}}$, $0 \leqslant i \leqslant \mathcal{B}(K)$
        \begin{example}[Intuition]    
            Quand l'automate est dans l'état $s_{\alpha,k}$, le clavier est dans l'état $s_\alpha$ avec un suffixe inutile de taille $k$. Quand il est en $\overline{s_{\alpha,k}}$, le clavier se trouve en $s_\alpha$, avec un suffixe inutile de taille $k$ \emph{et aucune lettre utile avant}. 
        \end{example}
        \item pour toute transition $s_{\alpha} \xrightarrow{\retour^r w} s_{\beta}$ de notre clavier à états, on crée dans $\A$ les transitions suivantes (voir figure \ref{ok}):
            \begin{enumerate}
                \item pour tout $r \leqslant k \leqslant \mathcal{B}(K) - \sharp t$, $s_{\alpha, k} \xrightarrow{\varepsilon} s_{\beta, k + \sharp t}$
                \item pour tout $0 \leqslant j < |w|$, $s_{\alpha, r} \xrightarrow{w[1,|w|-j]} s_{\beta, j}$
                \item pour tout $k \leqslant r$, $\overline{s_{\alpha,k}} \xrightarrow{\varepsilon} \overline{s_{\beta,|w|}}$
                \item pour tout $k > r$, $\overline{s_{\alpha,k}} \xrightarrow{\varepsilon} \overline{s_{\beta,k+\sharp t}}$
                \item pour tout $k \leqslant r$, pour tout $j < |w|$, $\overline{s_{\alpha,k}} \xrightarrow{w[1,|w|-j]} s_{\beta, j}$
            \end{enumerate} 
        %\item pour toute transition $init \xrightarrow{w} s_{\beta}$ du clavier, on ajoute, pour tout $j$ tel que $0 \leqslant j < |w|$, $\iota \xrightarrow{w[1,|w|-j]} s_{\beta, j}$, ainsi que $\iota \xrightarrow{\varepsilon} \bar{s_{\beta, |w|}}$. 
        \item l'état initial est $\overline{s_{0,0}}$, les états finaux sont les $s_{f,0}$ pour $f$ parcourant $F$.
    \end{itemize}
    
    
    Justifions informellement pourquoi notre ensemble de transitions est correct au regard de l'intuition donnée (la preuve formalisée est en annexe) :
    \begin{enumerate}
        \item Ces transitions ne font que modifier la taille du suffixe inutile. La contrainte $k \geqslant r$ pour appliquer $\retour^r w$ nous garantit qu'on n'efface pas de lettre utile (les autres conditions garantissent que les états considérés existent).
        \item Ces transitions écrivent des lettres utiles. Pour utiliser $\retour^r w$, il faut que la taille du suffixe inutile soit $r$. Ensuite, on écrit $w$, en disant qu'on écrit $|w|-j$ lettres utiles et $j$ inutiles.
        \item Ces transitions modifient le nombre de lettres inutiles : $\retour^r w$ efface les $k$ lettres inutiles qu'on avait, et essaie d'effacer plus (mais au bout d'un moment il n'y a plus de lettres à effacer), puis écrit $|w|$ lettres inutiles.
        \item Similaire au cas 1.
        \item Similaire aux cas 2. et 3.
    \end{enumerate}
    Remarquons qu'il n'existe aucune transition d'un état $s_{\alpha, k}$ vers un état $\bar{s_{\beta, k'}}$, ce qui correspond à l'intuition donnée plus tôt.
    
    \begin{bisimautomclav}\label{lklak}
        Pour $K$ un clavier de $\QRK$, on a $\L(K) = \L(\A(K))$.
    \end{bisimautomclav}
    \vspace{10pt}
    On peut donc affirmer :
    \begin{QRKegalRat}\label{th1}
        $\QRK = \Rat$
    \end{QRKegalRat}
    Le fait que $\mathcal{B}(K)$ soit polynomial en $|K|$ nous permet d'obtenir le corollaire comme quoi le problème du mot dans $\QRK$ est dans NP.
    \begin{thmbox}[L]{\textbf{\emph{Autre point de vue}}}
        Nous sommes partis des claviers de $\RK$, donc des machines qui écrivent et effacent, et nous leurs avons rajouté des états.
        En prenant le point de vue opposé, c'est-à-dire en partant d'automates (vus comme \emph{générateurs} et non \emph{accepteurs}), et en leur rajoutant le symbole $\retour$ qui annule la lettre écrite par la transition précédente, le théorème 1 peut se formuler comme ceci : \\
        \emph{Les automates pouvant effacer des lettres ont exactement la même expressivité que les automates habituels.}
    \end{thmbox}
    \begin{figure}
        \centering
        \begin{tikzcd}
            {s_{\alpha,0}}                                                                                                                                     &  &  & {s_{\beta,0}}              &  & {\overline{s_{\alpha,0}}} \arrow[rrrd, "\varepsilon", dashed]                                                                      &  &  & \vdots                       \\
            \vdots                                                                                                                                             &  &  & \vdots                     &  & \vdots                                                                                                                             &  &  & {\overline{s_{\beta,|w|}}}   \\
            {s_{\alpha,r}} \arrow[rrruu, "w", bend left] \arrow[rrru, "{w[1,|w|-j]}", dotted] \arrow[rrr, "\dots", dotted] \arrow[rrrd, "\varepsilon", dashed] &  &  & {}                         &  & {\overline{s_{\alpha,r}}} \arrow[rrru, "\varepsilon", dashed] \arrow[rrrdd, "w"] \arrow[rrrdddd, "{w[1]}"'] \arrow[rrrddd, dotted] &  &  & {\overline{s_{\beta,|w|+1}}} \\
            {s_{\alpha,r+1}} \arrow[rrrd, "\varepsilon", dashed]                                                                                               &  &  & {s_{\beta,|w|}}            &  & {\overline{s_{\alpha,r+1}}} \arrow[rrru, "\varepsilon", dashed]                                                                    &  &  &                              \\
            \vdots \arrow[rrrd, "\varepsilon", dotted]                                                                                                         &  &  & \vdots                     &  &                                                                                                                                    &  &  & {s_{\beta,0}}                \\
            \vdots \arrow[rrrd, "\varepsilon", dotted]                                                                                                         &  &  & \vdots                     &  &                                                                                                                                    &  &  & {}                           \\
            {s_{\alpha,\mathcal{B}(K)}}                                                                                                                        &  &  & {s_{\beta,\mathcal{B}(K)}} &  &                                                                                                                                    &  &  & {s_{\beta,|w|-1}}           
        \end{tikzcd}
        \caption{Les transitions ajoutées}
        \label{ok}
    \end{figure}
    \newpage


    \section{Claviers symétriques}
    Les claviers comme définis en section 1 sont en quelque sorte \emph{assymétriques} : même quand on peut se déplacer, on ne peut écrire et effacer qu'à gauche du curseur.
    On définit donc des opérations ajoutant de la symétrie.
    \begin{defopsym}[Opérations symétriques]
        Pour toute lettre $a \in A$, on définit les opérations d'écriture à gauche et à droite, $\egauche{a}$ et $\edroite{a}$, ainsi que l'effacement droit, noté $\suppr$.
        On définit leur sémantique :
        \begin{align*}
            \conf{u}{v} \cdot \egauche{a} &= \conf{ua}{v}   \\
            \conf{u}{v} \cdot \edroite{a} &= \conf{u}{av}   \\
            \conf{u}{av} \cdot \suppr &= \conf{u}{v}        \\
            \conf{u}{\varepsilon} \cdot \suppr &= \conf{u}{\varepsilon}
        \end{align*}
    \end{defopsym}
    On notera $\mathsf{B}X$ l'ensemble des claviers symétriques dont les transitions sont toutes de classe $X$ (à gauche ou à droite). On s'intéressera principalement à $\BK$ et $\BRK$.
    \subsection{Propriétés d'inclusion et de stabilité}
    Voilà quelques résultats sur les claviers symétriques.

    \begin{tabular}{|p{0.49\textwidth}|p{0.4\textwidth}|}
        \hline
        \textbf{Propriété} & \textbf{Exemple/Contre-exemple/Remarque} \\
        \hline
        $\BK \not\subset \Rat$ & $a^nb^n = \L({\egauche{a}\edroite{b}}) \in \BK$ \\
        \hline
        $\Rat \not\subset \BK$ & $a^*+b^* \notin \BK$ \\
        \hline
        $\BK$ n'est pas stable par union & \textit{idem} \\
        \hline
        $\BK$ et $\BRK$ sont stables par miroir & $\RK$ ne l'était pas, mais $\MK$ l'était \\
        \hline
        $\BK$ est exactement l'ensemble des langages générés par des grammaires linéaires à un seul terminal & à chaque touche $\egauche{w_g}\edroite{w_d}$ on associe la production $A \rightarrow w_gAw_d$ \\
        \hline
        $\mathsf{Lin} \subset \mathsf{QBK}$ & On crée un clavier à états avec un état par terminal, ainsi qu'un état final ajouté, et on transforme $A \rightarrow \alpha B \beta | \gamma$ en une touche $\overleftarrow{\alpha}\overrightarrow{\beta}$ de $s_A$ vers $s_B$, et une touche $\overleftarrow{\gamma}$ vers l'état final \\
        \hline
        La vacuité de l'intersection de deux langages de $\BRK$ est indécidable & \textit{cf.} annexe \\
        \hline
        $\BRK \not\subset \GRK$ & $(a^2)^*(b+b^2)$ \\
        \hline
    \end{tabular}

    
    Ce dernier langage ($(a^2)^*(b+b^2)$) nous permet aussi d'affirmer que les deux effacements sont nécessaires pour capturer toute l'expressivité de $\BRK$ ; il ne se trouve pas dans la classe sans effacement droit, et son miroir ne se trouve pas dans la classe sans effacement gauche.

    En effet, autant on peut simuler $\edroite{a}$ par $\egauche{a} \gauche$, autant $\suppr$ n'est pas toujours équivalent à $\droite\retour$ : elles n'ont pas le même effet sur la configuration $\conf{a}{\varepsilon}$.
         
    \subsection{Conjectures}\label{sectionclavsym}
    Les principaux exemples de langages de $\BRK$ non rationnels sont $a^nb^n$ (langage de $\{\egauche{a}\edroite{b}\}$), ou bien les palindromes (langage de $\{\egauche{a}\edroite{a},\egauche{b}\edroite{b}, \dots\}$). La similarité des claviers symétriques avec les grammaires linéaires nous poussent à penser que :
    \begin{example}[Conjecture 1]
        $\BRK \subset \Alg$
    \end{example}
    De plus, l'indécidabilité du caractère vide ou non de l'intersection de deux langages de $\BRK$ (voir annexe) nous montre que cette classe contient vraiment significativement plus que les rationnels, car ce problème est décidable sur ceux-ci.
    
    Cependant, même si on peut définir une notion de lettres inutiles (des deux côtés du curseur) similairement à ce qui est fait en section 2, la technique constituant à borner uniformément leur quantité pour construire une grammaire algébrique équivalente (simimairement au lemme \ref{bk}) ne fonctionne pas.
    Par exemple, étant donné le clavier $\{\egauche{a}\suppr,\retour\egauche{b}\}$, son langage est tout simplement $a^*b^*$, mais écrire le mot $a^kb^k$ nécessite $k$ lettres inutiles : cela consiste à écrire $a^{2k}$, puis remplacer $k$ $a$ par des $b$ (ou bien l'inverse). Mais dans tous les cas, le nombre de lettres reste constant dès qu'il y a des lettres des deux côtés. Donc il nous faut bien écrire $2k$ lettres d'un côté pour écrire $a^kb^k$.
    
    Ainsi, il n'existe aucune quantité ne dépendant que du clavier pouvant borner le nombre de lettres inutiles nécessaires à l'écriture d'un mot.
    
    \clearpage
    \appendix
    \section{Preuves}
    \begin{proof}[($\mathsf{QEK} = \mathsf{QK}$)]
        \emph{On montre que la touche entrée peut être négligée pour les claviers à états.} 

        Il suffit de créer, pour chaque état sur lequel arrive une transition finissant par $\entree$, une copie acceptante, vers laquelle on dirige cette transition (à laquelle on a enlevé $\entree$).
        Toutes les autres transitions vers l'état original y vont toujours, seules celles avec $\entree$ vont sur la copie acceptante. \medskip

        \begin{tikzpicture}[->,>=stealth',shorten >=1pt,auto,node distance=3.5cm,
            scale = 1,transform shape]
    
            \node[state] (s_0) {$s_0$};
            \node[state] (s_1) [left of=s_0] {$s_1$};
            \node[state] (s_1') [right of=s_0] {$s_1$};
            \node[state] (s_0') [right of=s_1'] {$s_0'$};
            \node[state,accepting] (s_{0,F}) [below of=s_0'] {$s_{0,F}$};

            
            \path (s_1) edge              node {$t_1t_2\cdots t_n \entree$} (s_0)
                    (s_1') edge              node {$t_1t_2\cdots t_n$} (s_{0,F});
        \end{tikzpicture}
    \end{proof}
    \begin{proof}[du lemme \ref{bk}]
        On abstrait par $\langle u,x,s \rangle$ les configurations de la forme $(w,s,\varepsilon)$ où $w = uv$ avec $|v| = x$ la taille du suffixe inutile\footnote{Cette notation rassemble donc plusieurs configurations ! L'intérêt est que le contenu du suffixe inutile ne change rien, seulement sa taille importe.}.
        
        On remarque un petit lemme : 
    \textit{
	Un calcul acceptant se décompose en une alternation de \textbf{pas}, définis comme une suite de transitions durant lesquelles le préfixe utile n'est pas modifié, et de transitions qui étendent strictement le préfixe utile.
	\[ \langle\varepsilon,0,s_0\rangle \rightsquigarrow \langle\varepsilon,8,\_\rangle \rightarrow \langle ab,5,\_\rangle \rightsquigarrow \langle ab,2,\_\rangle \rightarrow \langle abc,1,\_\rangle \rightsquigarrow \cdots \rightarrow \langle w,0,s_F\rangle \footnote{\_ désigne un état quelconque.}\]
	Les transitions utiles effacent toutes les lettres inutiles, puis ajoutent au moins une lettre utile et un certain nombre de lettres inutiles. Dans tous les cas le nombre de lettres inutiles avant et après leur application est borné par $\Kinf$. }

        Durant un pas, les lettres utiles (auxquelles on ne touche pas) ne nous intéressent pas, on ne s'intéresse qu'à aller dans le bon état avec un suffixe inutile de taille adaptée à la touche sur laquelle on veut appuyer.
        
        Il nous suffit donc de considérer que notre clavier est un système à un compteur, c'est à dire un ensemble d'états $Q'$ et un ensemble de transitions étiquetées par 0,1, ou -1, dans lequel une configuration est un élément de $Q' \times \varmathbb{N}$, et pour $(q,k,q')$ une transition, on peut passer de $(q,x)$ à $(q',x+k)$ tant que le compteur ne passe pas en dessous de 0.
        Ainsi, on transforme notre clavier en recopiant les états, mais en rajoutant à la place de toute transition $(q,\retour^rw,q')$, $r+|w|-1$ états : 
        
        \begin{tikzcd}
            q \arrow[r, "\retour^rw"] & q' \arrow[r, Rightarrow] & {} & q \arrow[r, "-1"] & q^t_1 \arrow[r, "-1", dotted] & q^t_r \arrow[r, "+1", dotted] & q^t_{|t|-1} \arrow[r, "+1"] & q'
        \end{tikzcd}

        Finalement, le nombre d'états de notre système est 
        \[n = |Q| + \sum_{t \in K} (|t| - 1) < |K|\] 
        Mais le premier pas, avant qu'il y aie des lettres utiles, doit être traité différemment, car on a le droit d'effacer plus de lettres inutiles que l'on a. 
        On utilise les tests à zéro : si on est encore en train d'effacer mais qu'on n'a plus de lettres inutiles, on peut "sauter" à l'endroit où on commence à écrire. On obtient ces transitions à la place de celles d'au-dessus :
        
        \begin{tikzcd}
            q \arrow[r, "-1"] \arrow[rr, bend right] & q^t_1 \arrow[r, "-1", dotted] \arrow[r, "=0 ?"', bend right] & q^t_r \arrow[r, "+1", dotted] & q^t_{|t|-1} \arrow[r, "+1"] & q'
        \end{tikzcd}
        
        Heureusement, on se trouve avec le même nombre d'états, ce qui ne changera pas la borne.

        Comme on se place dans un calcul acceptant, on considère que notre pas est possible : $(q_\alpha,x_\alpha) \rightarrow^* (q_\beta,x_\beta)$ ; on cherche à montrer que s'il existe un tel pas, alors il en existe un durant lequel $x$ est borné par $\mathcal{B}(K)$.
        Le théorème 2.2 de \cite{shortpathOCS} nous donne une borne de $14n^2 + n \max (x_\alpha,x_\beta)$ transitions empruntées. Or $x_\alpha,x_\beta \leq \Kinf$, et les transitions ne peuvent pas être toutes positives, car on doit revenir à une valeur de compteur en dessous de $\Kinf$. Le pire cas est donc $x_\alpha = x_\beta = \Kinf$, avec $14n^2+n\Kinf$ transitions empruntées, dont la moitié sont positives, ce qui amène le compteur à $(1+\frac{n}{2})\Kinf + 7n^2$ au plus.
        
        Comme $n,\Kinf < |K|$, $\mathcal{B}(K) = 9|K|^2$ convient.
       
    \end{proof}
    \begin{proof}[du lemme \ref{lklak}]
       Soit $v \in \L(\A(K))$. Il existe une exécution écrivant $v$ :
       \[ \overline{s_{0,0}} \rightarrow s_{k_1,n_1} \rightarrow \cdots \rightarrow s_{k_{|v|},0} \]
       par construction, on peut faire correspondre à chaque transition un appui sur une touche (non unique) :
       \begin{enumerate}
        \item si c'est une $\varepsilon$-transition entre $s_{s_i, n_i}$ et $s_{s_{i+1},n_{i+1}}$, alors il existe (dans $K$) $s_i \xrightarrow{\retour^{k}w} s_{i+1}$ avec $k \leqslant n_i$ et $|w| - r = n_{i+1} - n_i$
        \item si elle lit $j$ lettres de $v$ entre $s_{s_i,n_i}$ et $s_{s_{i+1},n_{i+1}}$, c'est qu'il existe $s_i \xrightarrow{\retour^{n_i}w} s_{i+1}$ avec $|w| = j + n_{i+1}$.
        \item si c'est une $\varepsilon$-transition entre $\bar{s_{s_i, n_i}}$ et $\bar{s_{s_{i+1},n_{i+1}}}$, alors il existe $s_i \xrightarrow{\retour^{k}w} s_{i+1}$ avec ou bien
            \begin{itemize}
                \item $k \geqslant n_i$ et $|w| = n_{i+1}$, ou
                \item $k < n_i$ et $|w| - r = n_{i+1} - n_i$
            \end{itemize}
        \item si elle lit $j$ lettres de $v$ entre $\bar{s_{s_i,n_i}}$ et $s_{s_{i+1},n_{i+1}}$, c'est qu'il existe $s_i \xrightarrow{\retour^{k}w} s_{i+1}$ avec $k \geqslant n_i$ et $|w| = j + n_{i+1}$
        %\item si c'est une $\varepsilon$-transition partant de $\iota$, elle arrive dans un certain $\bar{s_{s_1,n_1}}$, et c'est qu'il existait $init \xrightarrow{w} s_1$ avec $|w| = n_1$ après pré-traitement initial, donc qu'il existait $s_0 \xrightarrow{\retour^kw} s_1$ dans $K$.
        %\item si elle écrit $j$ lettres de $v$ entre $\iota$ et $s_{s_{1},n_{1}}$, c'est qu'il existait $init \xrightarrow{w} s_{1}$ avec $|w| = j + n_{1}$. après pré-traitement initial, donc qu'il existait $s_0 \xrightarrow{\retour^kw} s_1$ dans $K$.
       \end{enumerate}
       Le manque d'unicité peut par exemple venir du fait que plusieurs touches peuvent avoir $\sharp t_1 = \sharp t_2$, mais même si le mot écrit par les touches n'est pas le même, comme la différence se trouve dans les lettres inutiles, cela ne change rien.

       Dans tous les cas, on assure (par construction) que :
       \begin{itemize}
        \item quand l'automate lit des lettres, la touche correspondante écrit les mêmes lettres, et elles sont utiles (elle peut écrire des lettres inutiles en plus)
        \item on n'appuie jamais sur une touche qui effacerait des lettres utiles 
        \item quand pendant le calcul dans $\A(K)$ on est en $s_{s_i,k}$ avec $u$ déjà écrit, le calcul simulé de $K$ est en $\langle u,k,s_i\rangle$
        \item notre première transition part de $\overline{s_{0,0}}$, donc correspond bien à une touche partant de $s_0$ 
        \item notre exécution se termine dans un état final \textit{i.e.} de la forme $s_{f,0}, f \in F$
       \end{itemize} 
       Et donc, que $\A(K)$ simule effectivement $K$, c'est-à-dire que $\varepsilon,0,s_0 \rightarrow^*_K v,0,f$, donc $v \in \L(K)$, soit $\L(\A(K)) \subset \L(K)$.

       Le sens réciproque ne pose pas de problème particulier. On considère $w \in \L(K)$ et grâce au lemme \ref{bk}, on considère une exécution de $K$ qui reconnait $w$ en ayant toujours moins de $\mathcal{B}(K)$ lettres inutiles.
       A chaque appui sur une touche, il existe une seule transition dans l'automate correspondant à cette touche \emph{et au nombre de lettres inutiles restantes après}. L'invariant : Quand l'automate est dans l'état $s_{\alpha,k}$, le clavier est dans l'état $s_\alpha$ avec un suffixe inutile de taille $k$. Quand il est en $\overline{s_{\alpha,k}}$, le clavier se trouve en $s_\alpha$, avec un suffixe inutile de taille $k$ \emph{et aucune lettre utile avant}, est maintenu par construction.
    
       Finalement, $w \in \L(\A(K))$, donc $\L(K) = \L(\A(K))$.
    \end{proof}
    \begin{proof}[de l'indécidabilité de l'intersection dans $\BK$] 
        
        On réduit PCP.

        On se place sur un alphabet $A = A' \cup \{\diamond\}$, et on définit $f : A'^* \to A^*$ la fonction qui ajoute un $\diamond$ à droite de chaque lettre : $f(x_1x_2\dots x_n) = x_1\diamond x_2 \diamond \dots x_n \diamond$.
        On pose $K_p = \{\egauche{a}\edroite{a} | a \in A\}$, le clavier dont le langage est les palindromes pairs non vides.
        Soit $(u_i,v_i)_{1\leqslant i \leqslant n}$ une instance de PCP. On considère le clavier $K$ constitué des touches $t_i = \egauche{f(u_i)}\edroite{\widetilde{f(v_i)}}$.
        Une récurrence simple nous montre que 
        \[\conf{\varepsilon}{\varepsilon}\cdot(t_{i_1}...t_{i_k}) = \conf{f(u_{i_1})...f(u_{i_k})}{\widetilde{f(v_{i_k})}...\widetilde{f(v_{i_1})}}\]
        De plus, quand $k>0$, $f(u_{i_1})...f(u_{i_k})\widetilde{f(v_{i_k})}\dots\widetilde{f(v_{i_1})}$ contient un unique facteur $\diamond\diamond$, 
        entre la dernière lettre de $u_{i_k}$ et la première de $v_{i_k}$. Ce mot est donc un palindrome si, et seulement si
        $f(u_{i_1})...f(u_{i_k})$ est le miroir de $\widetilde{f(v_{i_k})}\dots\widetilde{f(v_{i_1})}$, ce qui équivaut à $u_{i_1}\dots u_{i_k} = v_{i_1} \dots v_{i_k}$.

        Ainsi, $\L(K) \cap \L(K_p) \neq \emptyset$ si, et seulement si, $(u_i,v_i)_{1\leqslant i \leqslant n} \in$ PCP, ce qui conclut.        
    \end{proof}
    \clearpage
    \printbibliography
\end{document}