\documentclass[11pt,french]{beamer}

\usepackage{newpxtext,newpxmath}
\usepackage[utf8]{inputenc}
\usepackage[T1]{fontenc}
\usepackage{mathtools,amssymb,stmaryrd}

\SetSymbolFont{stmry}{bold}{U}{stmry}{m}{n}
\usepackage{babel}
\newcommand*{\retour}{\mathord{\leftarrow}}
\newcommand*{\gauche}{\mathord{\blacktriangleleft}}
\newcommand*{\droite}{\mathord{\blacktriangleright}}
\newcommand*{\entree}{\mathord{\blacksquare}}

\newcommand*{\suppr}{\mathord{\rightarrow}}
\newcommand{\egauche}[1]{\overleftarrow{\mathtt{#1}}}
\newcommand{\edroite}[1]{\overrightarrow{\mathtt{#1}}}


\newcommand*{\Clav}{\ensuremath{\mathsf{Clav}}}
\newcommand*{\Rat}{\ensuremath{\mathsf{Rat}}}
\newcommand*{\Rec}{\ensuremath{\mathsf{Rec}}}
\newcommand*{\Alg}{\ensuremath{\mathsf{Alg}}}
\newcommand*{\Cont}{\ensuremath{\mathsf{Cont}}}

\newcommand*{\MK}{\ensuremath{\mathsf{MK}}}

\newcommand*{\EK}{\ensuremath{\mathsf{EK}}}
\newcommand*{\RK}{\ensuremath{\mathsf{RK}}}
\newcommand*{\REK}{\ensuremath{\mathsf{REK}}}

\newcommand*{\FK}{\ensuremath{\mathsf{FK}}}
\newcommand*{\FEK}{\ensuremath{\mathsf{FEK}}}
\newcommand*{\FRK}{\ensuremath{\mathsf{FRK}}}

\newcommand*{\GK}{\ensuremath{\mathsf{GK}}}
\newcommand*{\GRK}{\ensuremath{\mathsf{GRK}}}
\newcommand*{\GEK}{\ensuremath{\mathsf{GEK}}}

\newcommand*{\FREK}{\ensuremath{\mathsf{FREK}}}
\newcommand*{\GREK}{\ensuremath{\mathsf{GREK}}}
\newcommand*{\BREK}{\ensuremath{\mathsf{BREK}}}

\newcommand*{\QMK}{\ensuremath{\mathsf{QMK}}}
\newcommand*{\QRK}{\ensuremath{\mathsf{QRK}}}
\newcommand*{\QGRK}{\ensuremath{\mathsf{QGRK}}}
\newcommand*{\QFRK}{\ensuremath{\mathsf{QFRK}}}

\newcommand*{\BK}{\ensuremath{\mathsf{BK}}}
\newcommand*{\BRK}{\ensuremath{\mathsf{BRK}}}

\let\touche\mathtt
\let\key\touche

\newcommand*{\conf}[2]{
	\left\langle #1 \middle| #2 \right\rangle
}
\let\config\conf


\newcommand*{\act}[1]{\xrightarrow{#1}}
\newcommand*{\actEff}[1]{\xrightarrow{#1}_{\text{e}}}
\newcommand*{\cdotEff}{\odot} %{\cdot_{\text{e}}}
\usetheme{metropolis}
\metroset{progressbar=frametitle}
\metroset{block=fill}

\usepackage{tikz}     % Un package pour les dessins
\usepackage{tikz-cd}
\usetikzlibrary{cd, automata,positioning,arrows}
\usepackage{adjustbox}

\renewcommand{\L}{\mathcal{L}}
\renewcommand{\bar}{\overline}
\newcommand{\A}{\mathcal{A}}
\newcommand{\Kinf}{||K||_{\infty}}


\title{Claviers symétriques et à états}
\author[Erwann LOULERGUE]{Erwann LOULERGUE, \texorpdfstring{\\supervisé par Vincent PENELLE et Corto MASCLE}{}}
\date{Juin \& Juillet 2022}
\institute[ENS Paris-Saclay]{Au LaBRI}


\begin{document}

\begin{frame}
	\titlepage
\end{frame}

\begin{frame}{Sommaire}
	\tableofcontents
\end{frame}

\section{Les claviers "habituels"}
\begin{frame}
	\frametitle{Le contexte}
	\begin{center}
		\includegraphics[height=0.3\textwidth]{Drame.jpg}	
	\end{center}

	On possède un clavier ne fonctionnant pas normalement. Par exemple, si on appuie sur la touche $A$, il écrit $\touche{abc}$, si on appuie sur la touche $B$, il efface deux caractères\dots

	Quels mots peut-on écrire avec ce clavier ? Et de manière générale, étant donné un clavier, peut-on écrire un mot donné ?

\end{frame}
\begin{frame}{Définitions}
	\begin{block}{opérations élementaires}
		\begin{itemize}
			\item $\touche{a}$ pour $a \in \Sigma$: écrit "a" à gauche du curseur. \pause
			\item $\retour$: efface la lettre à gauche du curseur. \pause
			\item $\gauche$ et $\droite$: déplace le curseur à gauche ou à droite.
		\end{itemize}
	\end{block}
	\vspace{-8pt}
	On note $S$ l'ensemble des opérations élémentaires.
	\pause
	\begin{block}{Claviers}
		\begin{itemize}
			\item Une \emph{touche} est une suite d'opérations élémentaires. \pause
			\item Un \emph{clavier (automatique)} est un ensemble fini de touches.
		\end{itemize}
	\end{block}
	\begin{exampleblock}{Le clavier précédent}
		Par exemple, le clavier discuté précedemment est $\{\touche{abc},\retour^2\}$
	\end{exampleblock}
\end{frame}

\begin{frame}{Modélisation}
	Quand le mot courant est $uv$ avec le curseur entre $u$ et $v$, on note la configuration $\conf{u}{v}$. \pause

	Les opérations élémentaires induisent des actions sur les configurations :
	\begin{align*}
		\forall a \in A&, \conf{u}{v}\cdot a = \conf{ua}{v} \\
		\conf{\varepsilon}{v}\cdot \retour = \conf{\varepsilon}{v} &, \conf{ua}{v}\cdot \retour = \conf{u}{v} \\
		\conf{\varepsilon}{v}\cdot \gauche = \conf{\varepsilon}{v} &, \conf{ua}{v}\cdot \gauche = \conf{u}{av} \\
		\conf{u}{\varepsilon}\cdot \droite = \conf{u}{\varepsilon} &, \conf{u}{av}\cdot \droite = \conf{ua}{v} \\
	\end{align*}
\end{frame}

\begin{frame}{Modélisation}
	\begin{exampleblock}{Action d'une touche}
		On applique $t = \retour a \droite$ à $\config{c}{d}$. \pause
		 \begin{align*}
			\config{c}{d} & \act{\retour} \config{\varepsilon}{d}\\
						  & \act{a}       \config{a}{d}\\
						  & \act{\droite} \config{ad}{\varepsilon}.
		\end{align*}
		Ainsi $\conf{c}{d}\cdot t = \conf{ad}{\varepsilon}$.
	\end{exampleblock}
	
\end{frame}

\begin{frame}{Les \emph{vrais} claviers} 
	Ces claviers sont assez limités : on n'a pas de notion de saisie qu'on choisirait d'arrêter. \pause On définit donc les \emph{claviers manuels} :
    \begin{block}{Clavier (manuel)}
        Un clavier manuel, ou simplement \textbf{clavier}, est un couple $(T,F)$ d'ensembles finis de touches.
        Une exécution acceptante d'un clavier est une suite de touches de $T$ suivie d'une touche de $F$.
    \end{block} \pause
	\begin{exampleblock}{Remarque}
		Le fait qu'une touche soit finale peut être modélisé par une touche entrée qu'on notera $\entree$.
	\end{exampleblock}
\end{frame}

\begin{frame}{Le langage d'un clavier}
	Pour les claviers automatiques, on a \[\L(K) = \{w | w =uv, \exists t_1...t_n, \conf{\varepsilon}{\varepsilon}\cdot t_1...t_n = \conf{u}{v}\}\] \pause
    Sinon, dans le cas général, on définit $\L(K)$ comme \[\{w | w =uv, \exists t_1...t_n, \conf{\varepsilon}{\varepsilon}\cdot t_1...t_n = \conf{u}{v}, \mathrm{seule~} t_n \mathrm{~est~finale}\}\]
\end{frame}


\begin{frame}
	\frametitle{Classes de claviers}
	\[
	 \begin{array}{cc}
		\mathsf{R}:  \retour & \mathsf{E}:  \entree \\
        \mathsf{G}:  \gauche & \mathsf{F}:  \droite + \gauche \\
	 \end{array}
	\]

	\[
		\begin{array}{ccc}
			\MK : \{\} & \GK : \{\gauche\} & \FK :\{\gauche,\droite\} \\
			\EK : \{\entree\} & \GEK : \{\gauche,\entree\} & \FEK :\{\gauche,\droite,\entree\} \\
			\RK : \{\retour\} & \GRK : \{\gauche,\retour\} & \FRK :\{\gauche,\droite,\retour\} \\
			\REK : \{\retour, \entree\} & \GREK : \{\gauche,\retour,\entree\} & \FREK :\{\gauche,\droite,\retour,\entree\} \\
		\end{array}
	\]
 \end{frame}


\section{Claviers à états}
\begin{frame}{Définition}
	\begin{block}{Clavier à état}
		Un clavier à états est la donnée de $(Q,\Delta ,F,s_0)$ où :
        \begin{itemize}
            \item $Q$ est un ensemble d'\textbf{états}
            \item $\Delta \subset Q \times S^* \times Q$ est un ensemble fini de \textbf{transitions}
            \item $F$ est un ensemble d'états dits \textbf{finaux}
            \item $s_0 \in Q$ est un état dit \textbf{initial}
        \end{itemize}
	\end{block}
	\pause
	\begin{block}{Configuration étatique}
		La configuration d'un clavier à états pendant son calcul est décrite par $(u,q,v) \in A^* \times Q \times A^*$, appelé \textbf{configuration étatique}.
	\end{block}
\end{frame}
\begin{frame}{Définition}
	\begin{exampleblock}{Action d'une touche sur une configuration}	
		Soit $(u,q,v)$ une configuration étatique, et $(q,t,q') \in \Delta$. En appliquant cette transition, on passe dans la configuration $(u',q',v')$, avec $\conf{u'}{v'} = \conf{u}{v} \cdot t$.
	\end{exampleblock}
	Une touche modifie le mot (clavier) et fait passer dans un autre état (automate).
	\pause
    On peut donc définir le langage d'un clavier à états :
    \[ \{uv | \exists\delta_1,...,\delta_n \in \Delta, \exists s_f \in F, (u,s_f,v) = (\varepsilon,s_0,\varepsilon)\cdot\delta_1\cdots\delta_n \}\]
	\pause
    On notera $\mathsf{Q}X$ l'ensemble des claviers à états dont les transitions sont toutes de classe $X$ : $\QMK$, $\QFRK$, \textit{etc}... 
\end{frame}
\begin{frame}{$\QRK = \Rat$}
	$\QRK$ est un bon candidat pour être égal à $\Rat$, car il ajoute des états à $\RK$.
	Une inclusion est triviale.
	\pause

	On veut donc montrer que pour tout clavier $K \in \QRK$, il existe un automate fini le simulant.
	\pause

	On sait que dans un calcul de $K$, les configurations sont toutes de la forme $(w,q,\varepsilon)$. On divise le mot $w$ en deux :
	\begin{itemize}
		\item un préfixe \emph{utile} qui ne sera jamais effacé.
		\item un suffixe \emph{inutile} qui sera effacé pendant l'éxécution.
	\end{itemize}
\end{frame}
\begin{frame}{Le lemme clé}
	On va essayer de construire un automate $\A(K)$ dont les états seront $(s,k)$ pour $s$ un état de $K$, et $k$ la taille du suffixe inutile : l'automate lit une lettre quand $K$ écrit une lettre \emph{utile}.
	\pause

	\emph{A priori}, ces états ne sont pas en nombre fini\dots 
	\pause
	\begin{alertblock}{Lemme}
		Soit $K$ un clavier à états de $\QRK$. Il existe un entier $\mathcal{B}(K)$, ne dépendant que de $K$, tel que pour tout mot $w \in \L(K)$,
        il existe une exécution de $K$ qui écrit $w$ et dont le suffixe inutile est de longueur majorée par $\mathcal{B}(K)$ à chaque étape. \\
        De plus, $\mathcal{B}(K)$ est polynomial en la taille de $K$.
	\end{alertblock}
\end{frame}

\begin{frame}{La construction}
	\adjustbox{scale=0.6,center}{%
		$$
			\begin{tikzcd}[ampersand replacement=\&]
				{s_{\alpha,0}}                                                                                                                                     \&  \&  \& {s_{\beta,0}}              \&  \& {\overline{s_{\alpha,0}}} \arrow[rrrd, "\varepsilon", dashed]                                                                      \&  \&  \& \vdots                       \\
				\vdots                                                                                                                                             \&  \&  \& \vdots                     \&  \& \vdots                                                                                                                             \&  \&  \& {\overline{s_{\beta,|w|}}}   \\
				{s_{\alpha,r}} \arrow[rrruu, "w", bend left] \arrow[rrru, "{w[1,|w|-j]}", dotted] \arrow[rrr, "\dots", dotted] \arrow[rrrd, "\varepsilon", dashed] \&  \&  \& {}                         \&  \& {\overline{s_{\alpha,r}}} \arrow[rrru, "\varepsilon", dashed] \arrow[rrrdd, "w"] \arrow[rrrdddd, "{w[1]}"'] \arrow[rrrddd, dotted] \&  \&  \& {\overline{s_{\beta,|w|+1}}} \\
				{s_{\alpha,r+1}} \arrow[rrrd, "\varepsilon", dashed]                                                                                               \&  \&  \& {s_{\beta,|w|}}            \&  \& {\overline{s_{\alpha,r+1}}} \arrow[rrru, "\varepsilon", dashed]                                                                    \&  \&  \&                              \\
				\vdots \arrow[rrrd, "\varepsilon", dotted]                                                                                                         \&  \&  \& \vdots                     \&  \&                                                                                                                                    \&  \&  \& {s_{\beta,0}}                \\
				\vdots \arrow[rrrd, "\varepsilon", dotted]                                                                                                         \&  \&  \& \vdots                     \&  \&                                                                                                                                    \&  \&  \& {}                           \\
				{s_{\alpha,\mathcal{B}(K)}}                                                                                                                        \&  \&  \& {s_{\beta,\mathcal{B}(K)}} \&  \&                                                                                                                                    \&  \&  \& {s_{\beta,|w|-1}}           
			\end{tikzcd}
		$$
	}
	\pause
	\begin{center}
		Cet automate reconnait le même langage que $K$.
	\end{center}
\end{frame}
\begin{frame}{Qu'a-t-on vraiment prouvé ?}
	Nous sommes partis des claviers de $\RK$, donc des machines qui écrivent et effacent, et nous leurs avons rajouté des états.
	\pause

    En prenant le point de vue opposé (en partant des automates), notre résultat peut se formuler comme ceci : \\
    \emph{Les automates pouvant effacer des lettres ont exactement la même expressivité que les automates habituels.}
\end{frame}

\section{Claviers symétriques}
\begin{frame}{Définition}
	

\end{frame}
\end{document}