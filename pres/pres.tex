\documentclass[french]{beamer}

\usepackage[utf8]{inputenc}
\usepackage[T1]{fontenc}
\usepackage{lmodern}
\usepackage{amsmath, amssymb}

\usepackage{babel}
\newcommand*{\retour}{\mathord{\leftarrow}}
\newcommand*{\gauche}{\mathord{\blacktriangleleft}}
\newcommand*{\droite}{\mathord{\blacktriangleright}}
\newcommand*{\entree}{\mathord{\blacksquare}}

\newcommand*{\suppr}{\mathord{\rightarrow}}
\newcommand{\egauche}[1]{\overleftarrow{\mathtt{#1}}}
\newcommand{\edroite}[1]{\overrightarrow{\mathtt{#1}}}


\newcommand*{\Clav}{\ensuremath{\mathsf{Clav}}}
\newcommand*{\Rat}{\ensuremath{\mathsf{Rat}}}
\newcommand*{\Rec}{\ensuremath{\mathsf{Rec}}}
\newcommand*{\Alg}{\ensuremath{\mathsf{Alg}}}
\newcommand*{\Cont}{\ensuremath{\mathsf{Cont}}}

\newcommand*{\MK}{\ensuremath{\mathsf{MK}}}

\newcommand*{\EK}{\ensuremath{\mathsf{EK}}}
\newcommand*{\RK}{\ensuremath{\mathsf{RK}}}
\newcommand*{\REK}{\ensuremath{\mathsf{REK}}}

\newcommand*{\FK}{\ensuremath{\mathsf{FK}}}
\newcommand*{\FEK}{\ensuremath{\mathsf{FEK}}}
\newcommand*{\FRK}{\ensuremath{\mathsf{FRK}}}

\newcommand*{\GK}{\ensuremath{\mathsf{GK}}}
\newcommand*{\GRK}{\ensuremath{\mathsf{GRK}}}
\newcommand*{\GEK}{\ensuremath{\mathsf{GEK}}}

\newcommand*{\FREK}{\ensuremath{\mathsf{FREK}}}
\newcommand*{\GREK}{\ensuremath{\mathsf{GREK}}}
\newcommand*{\BREK}{\ensuremath{\mathsf{BREK}}}

\newcommand*{\QMK}{\ensuremath{\mathsf{QMK}}}
\newcommand*{\QRK}{\ensuremath{\mathsf{QRK}}}
\newcommand*{\QGRK}{\ensuremath{\mathsf{QGRK}}}
\newcommand*{\QFRK}{\ensuremath{\mathsf{QFRK}}}

\newcommand*{\BK}{\ensuremath{\mathsf{BK}}}
\newcommand*{\BRK}{\ensuremath{\mathsf{BRK}}}

\let\touche\mathtt
\let\key\touche

\newcommand*{\conf}[2]{
	\left\langle #1 \middle| #2 \right\rangle
}
\let\config\conf


\newcommand*{\act}[1]{\xrightarrow{#1}}
\newcommand*{\actEff}[1]{\xrightarrow{#1}_{\text{e}}}
\newcommand*{\cdotEff}{\odot} %{\cdot_{\text{e}}}
\usetheme{boadilla}


%Pour le TITLEPAGE
\title{Claviers symétriques et à états}
\author[Erwann LOULERGUE]{Erwann LOULERGUE, \texorpdfstring{\\supervisé par Vincent PENELLE et Corto MASCLE}{}}
\date{Juin \& Juillet 2022}
\institute[ENS Paris-Saclay]{Au LaBRI}


\begin{document}

\begin{frame}
	\titlepage
\end{frame}

\begin{frame}{Sommaire}
	\tableofcontents
\end{frame}

\section{Définitions}
\begin{frame}{Opérations élémentaires}
	blabla
\end{frame}

\begin{frame}
	Un \textbf<2,3>{texte} en gras. 
	\visible<3>{Un texte visible sur la 3\ieme{} couche}
\end{frame}

\begin{frame}{Titre (facultatif)} 
\framesubtitle{Sous titre (facultatif aussi)}
	\begin{block}{Remarque}
	Un bloc
	\end{block}
	
	\begin{alertblock}{Proposition}
	Un bloc alerte
	\end{alertblock}
	
	\begin{exampleblock}<2>{Exemple}
	Un bloc exemple qui est visible sur la 2\ieme{} couche : $f(x)=2x$.
	\end{exampleblock}
\end{frame}

\section{Claviers à états}
\begin{frame}{La section 2 commence}
\begin{itemize}
	\item<1-> On peut cliquer sur les titres de la barre de gauche pour naviguer dans les sections du pdf (essayez !).
	\item<2-> On peut changer le ``look'' du beamer, en changeant de thème. Retournez dans le fichier source et compilez avec les autres thèmes proposés (il existe énormément de thèmes; seuls trois sont proposés dans le source).
\end{itemize}

\end{frame}

\end{document}